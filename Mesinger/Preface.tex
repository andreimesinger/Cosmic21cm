\chapter*{Preface}
\label{Preface}


The Cosmic Microwave Background (CMB) gives us a remarkable image of the Universe when it was just $\sim$ 400 000 years old (less than 3\% of its current age).  At that time, the normal (atomic) matter in our Universe recombined, allowing it to separate from the CMB, and begin to collapse under gravity.

However, the billion years that followed this recombination epoch are still mostly shrouded in darkness.  Although observations remain sparse, we know they must have witnessed the birth of the very first stars, black holes, and galaxies.  The light from these nascent objects spread out, heating and ionizing virtually all of the atoms in existence.  This epoch of reionization was the final major phase transition of our Universe, and the last interesting thing to happen to most of the atoms.

As a graduate student in 2003, I remember the palpable excitement in anticipation of the first measurement of the optical depth to the CMB from the Wilkinson Microwave Anisotropy Probe ({\it WMAP}).  Prior to this, we only had evidence that the Universe was largely ionized up to $z<$5--6, but had almost no clue about when this reionization actually happened.  The optical depth estimate turned out to be large, stimulating a flury of research papers on early reionization by exotic, unseen sources.  Although subsequent measurements brought down the optical depth, the following decade and a half witnessed a surge of activity in inferring the ionization state of the Universe
from observations of high-$z$ quasars, galaxies and the CMB.  Thanks to sophisticated observational and analysis techniques, astronomers were able to squeeze out estimates on the timing of reionization from a fairly modest amount of data.
The emerging picture is that the bulk of reionization occurred around $z\sim$7--8, driven by galaxies too faint to be observed directly.

But what else can we learn about the first billion years?  They comprise the bulk of our past light-cone.  The number of independent modes in this light-cone is orders of magnitude larger than that in the CMB.  If we could tap into this vast resource, we could unlock the mysteries of how the first stars and galaxies formed, how they interacted with each other, and open up a new window for physical cosmology.

Thankfully, we have a tool to do just that: {\it the cosmic 21-cm signal}.  Corresponding to the spin-flip transition of neutral hydrogen, the 21-cm line is sensitive to the temperature and ionization state of the cosmic gas, as well as to cosmological parameters.  It is a line transition, so different observed frequencies correspond to different redshifts.  Therefore upcoming interferometers will allow us to {\it map out the first billion years of our Universe!}  The patterns of this map will tell us about the properties of the unseen first generations of galaxies, provided we know how to interpret them.  Cosmic dawn and reionization will move from being observationally starved epochs to being at the frontier of Big Data analysis.

We are truly at the cusp of a revolution.  Thankfully not a violent one, but one that can transform our understanding of the Universe in which we live.  I hope that this book can help convince you to join the revolution!
